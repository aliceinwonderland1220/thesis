\chapter{Data and MC samples}
\label{ch:ana-intro}

\section{Data samples}

\par The proton-proton collision data (Good-Run-Lists) are considered as good when both the LHC and the ATLAS detector are running in a stable and healthy way.
On the ATLAS side, the detector system is considered healthy when all sub-detectors, central trigger and data acquisition system are reported healthy in data quality monitoring system and certified by data quality group. 
The proton-proton collisions at a centre-of-mass energy of 13~\TeV~recorded between 2015 and 2017 are used in this search, and the corresponding Good-Run-Lists are showed below: 

\begin{itemize}
	\tiny
	\item \texttt{data15\_13TeV.periodAllYear\_DetStatus-v89-pro21-02\_Unknown\_PHYS\_StandardGRL\_All\_Good\_25ns.xml}
	\item \texttt{data16\_13TeV.periodAllYear\_DetStatus-v89-pro21-01\_DQDefects-00-02-04\_PHYS\_StandardGRL\_All\_Good\_25ns.xml}
	\item \texttt{data17\_13TeV.periodAllYear\_DetStatus-v99-pro22-01\_Unknown\_PHYS\_StandardGRL\_All\_Good\_25ns\_Triggerno17e33prim.xml}
	\item \texttt{data18\_13TeV.periodAllYear\_DetStatus-v102-pro22-04\_Unknown\_PHYS\_StandardGRL\_All\_Good\_25ns\_Triggerno17e33prim.xml}
\end{itemize}

\par The resulting dataset corresponds to integrated luminosities of 3.2~\ifb, 33.0~\ifb, 44.3~\ifb\ and 58.5~\ifb\ for each data taking year, respectively. 
The total integrated luminosity is 139.0~\ifb.

\section{MC samples}
\label{ch:data-mc-samples}

\subsection{2HDM+a and Z'-2HDM signal}
\label{subsec:signal}

\par Simulated events corresponding to the $pp\to h\chi\bar{\chi}$ process are generated with \textsc{MadGraph5\_aMC@NLO} (\textsc{MG5\_aMC}) 2.6.1~\cite{Alwall:2014hca} 
based on the Universal FeynRules Output (UFO) model developed in Ref~\cite{Abe:2018bpo} at leading-order (LO) accuracy using the NNPDF 3.0 next-to-leading order~(NLO) PDF set with $\alpha_s=0.118$~\cite{Ball:2014uwa}. 
The signal samples are generated separately for the loop-induced gluon--gluon fusion (ggF) process and the $b$-initiated production. 

\par Simulated events corresponding to the $pp\to Z'\to h A(\chi\bar{\chi})$ process are generated with \textsc{MadGraph5\_aMC@NLO} (\textsc{MG5\_aMC}) 2.2.3~\cite{Alwall:2014hca} 
at leading-order (LO) accuracy using the NNPDF 2.3 NLO PDF set with $\alpha_s=0.119$~\cite{Ball:2012cx}. 
The 4-flavour scheme is used for the calculation of the matrix elements. Several samples are generated for different values of $m(Z')$ and $m(A)$.
The masses of the additional Higgs bosons are fixed to $m(H)=m(H^{\pm})=300$~\GeV~and the mass of the dark matter candidates is fixed to $m(\chi)=100$~\GeV. 

\par The parton shower and hadronisation are simulated with \textsc{Pythia} 8.230~\cite{Sjostrand:2014zea} and \textsc{Pythia} 8.186~\cite{Sjostrand:2007gs}, respectively for 2HDM+a and Z' 2HDM signal,
using the A14 set~\cite{ATL-PHYS-PUB-2014-021} of tuned parameters together with the NNPDF 2.3 LO PDF set~\cite{Ball:2011mu}. 
Higgs boson decays into $b\bar{b}$ pairs are also simulated with according \textsc{Pythia} versions with a branching fraction fixed to the SM prediction.

\subsection{V+jets}

\par $V+$ jets production is simulated with the \textsc{Sherpa} v2.2~\cite{Bothmann:2019yzt} generator. 
In this setup, NLO-accurate matrix elements for up to two jets, and LO-accurate matrix elements for up to four jets are calculated with the Comix~\cite{Gleisberg:2008fv} and OpenLoops~\cite{Cascioli:2011va,Denner:2016kdg} libraries. 
They are matched with the \textsc{Sherpa} parton shower~\cite{Schumann:2007mg} using the MEPS@NLO prescription~\cite{Hoeche:2011fd,Hoeche:2012yf,Catani:2001cc,Hoeche:2009rj}.
Samples are generated using the \nnpdfnnlo set~\cite{Ball:2014uwa}.

\par The samples are split according to whether they contain a $B$ hadron or no $B$ with \pt~$>$ 5~\GeV~and $|\eta| < 2.9$, 
a $C$ hadron with \pt~$>$ 4~\GeV~and $|\eta| < 3$ (with the filtered samples called Bfilter, CFilterBVeto, CVetoBVeto respectively). 
They are further split by either:

\begin{itemize}
	\item using the transverse momentum of the $V$ boson produced by Sherpa ($p_T(V)$), or
	\item using the $\max(p_T(V), H_T)$, where $\ensuremath{H_{\mathrm{T}}}$ is the scalar sum of the \pt~of the vector boson and the jets.
\end{itemize}%

\subsection{$t\bar{t}$}

\par The production of \ttbar~events are generated using the \powhegbox~\cite{Frixione:2007nw,Nason:2004rx,Frixione:2007vw,Alioli:2010xd}~v2
generator which provides matrix elements at NLO with the NNPDF3.0NLO~\cite{Ball:2014uwa} parton distribution function (PDF) and the \hdamp\ parameter\footnote{The \hdamp\ parameter
controls the transverse momentum \pt\ of the first additional emission beyond the leading-order Feynman diagram
in the parton shower and therefore regulates the  high-\pt\ emission against which the \ttbar\ system recoils.} set to 1.5~\mtop~\cite{ATL-PHYS-PUB-2016-020}.
The functional form of the renormalisation and factorisation scale is set to the default scale $\sqrt{m_{\textrm{top}}^2 + p_{\textrm T}^2}$.
The events are interfaced with \pythia~\cite{Sjostrand:2014zea} for the parton shower and hadronisation,
using the A14 set of tuned parameters~\cite{ATL-PHYS-PUB-2014-021}  and the NNPDF23LO PDF set.
The decays of bottom and charm hadrons are simulated using the \evtgen\ v1.6.0 program~\cite{EvtGen}.

\par The events are filtered to select di-leptonic or semi-leptonic $t\bar{t}$ decays. Two sets of samples are used in the analysis

\begin{itemize}
	\item an inclusive $t\bar{t}$ sample, and
	\item a set of \met-filtered $t\bar{t}$ samples, which is specifically developed to reduce the MC statistical uncertainties in the high-\met~regions.
\end{itemize}

\subsection{Single top}

\par Single-top $tW$ associated production, single-top t-channel and s-channel productions are all modelled using the \powhegbox~\cite{Frederix:2012dh,Nason:2004rx,Frixione:2007vw,Alioli:2010xd}~v2
generator which provides matrix elements at NLO\
with the NNPDF3.0NLOnf4~\cite{Ball:2014uwa} PDF set.
The diagram removal scheme~\cite{Frixione:2008yi} is employed to handle the interference with \ttbar\ production~\cite{ATL-PHYS-PUB-2016-020}.
The events are interfaced with \pythia.230~\cite{Sjostrand:2014zea} using the A14 tune~\cite{ATL-PHYS-PUB-2014-021} and the NNPDF23LO PDF set.
The decays of bottom and charm hadrons are simulated using the \evtgen\ v1.6.0 program~\cite{EvtGen}.
For the t-channel, the functional form of the renormalisation and factorisation scale is set to $\sqrt{m_{\textrm{b}}^2 + p_{{\textrm T},b}^2}$
following the recommendation of Ref~\cite{Frederix:2012dh} while for the other two, the functional form of the renormalisation and factorisation scale is set to the default scale, which is equal to the top quark mass.

\subsection{Diboson}

\par Diboson samples are simulated with the \sherpa~v2.2~\cite{Bothmann:2019yzt} generator. 
In this setup multiple matrix elements are matched and merged with the \sherpa parton shower based on Catani-Seymour dipole~\cite{Gleisberg:2008fv,Schumann:2007mg} using the MEPS@NLO prescription~\cite{Hoeche:2011fd,Hoeche:2012yf,Catani:2001cc,Hoeche:2009rj}. 
The virtual QCD correction for matrix elements at NLO accuracy are provided by the \openloops\ library~\cite{Cascioli:2011va,Denner:2016kdg}. 
Samples are generated using the \nnpdfnnlo set~\cite{Ball:2014uwa}.

\subsection{SM $Vh(b\bar{b})$}

\par The production of $q\bar{q}\to Wh\to \ell\nu b\bar{b}$ and $q\bar{q}\to Zh\to \ell\ell b\bar{b}$ events is modelled using the \powhegbox v2 generator
~\cite{Alioli:2010xd} using the \minlo procedure~\cite{Hamilton:2012np,Luisoni:2013kna} with the NNPDF3.0NLO~\cite{Ball:2014uwa} PDF set.
The events are interfaced to \pythia~\cite{Sjostrand:2014zea}~ using the AZNLO tune~\cite{Aad:2014xaa} and the CTEQ6L1~\cite{Pumplin:2002vw} PDF set.

\par The loop-induced $gg\to Zh \to \ell b\bar{b}, \nu\bar{\nu}b\bar{b}$ process is modelled using the \powhegbox v2 generator~\cite{Alioli:2010xd} with the NNPDF3.0NLO~\cite{Ball:2014uwa} PDF set.
Parton showering and hadronisation are provided by \pythia~\cite{Sjostrand:2014zea} with the same settings as the one used for the $q\bar{q}$ process.

\subsection{$t\bar{t}+Z/H$}

\par \ttH~events are modelled using the \powhegbox~\cite{Frixione:2007nw,Nason:2004rx,Frixione:2007vw,Alioli:2010xd,Hartanto:2015uka}
generator at NLO with the NNPDF3.0NLO~\cite{Ball:2014uwa} PDF set.
The events are interfaced with \pythia~\cite{Sjostrand:2014zea}~ using the A14 tune~\cite{ATL-PHYS-PUB-2014-021} and the NNPDF2.3LO~\cite{Ball:2014uwa} PDF set.

\par \ttV~events are modelled using the \mgamc~v2.3.3~\cite{Alwall:2014hca} generator which provides matrix elements at next-to-leading order~(NLO) with the NNPDF3.0NLO~\cite{Ball:2014uwa} parton distribution function~(PDF).
The functional form of the renormalization and factorization scale is set to the default scale 0.5$\times \sum_i \sqrt{m^2_i+p^2_{T,i}}$.
Top quarks are decayed at LO using \madspin~\cite{Frixione:2007zp,Artoisenet:2012st} to preserve all spin correlations.
The events are interfaced with \pythia~\cite{Sjostrand:2014zea} for the parton shower and hadronisation,
using the A14 set of tuned parameters~\cite{ATL-PHYS-PUB-2014-021}  and the NNPDF23LO~\cite{Ball:2014uwa} PDF set.
The decays of bottom and charm hadrons are simulated using the \evtgen\ v1.2.0 program~\cite{EvtGen}.

\section{Trigger}
\label{sec:trigger}

\par In this analysis, we are looking at the signal events with a Higgs boson, which can be decayed into $b$-anti-$b$ pair with large \met. 
Therefore, \met~trigger effiencies are important to evaluate yields in signal region. 
Moreover, we need to estimate background in the leptonic control regions, and therefore single lepton trigger efficiencies also needed to be considered.

\par \MET~triggers are used for this he zero- and one-lepton regions and unprescaled single-muon triggers in the two-lepton region. 
Their thresholds are determined by requiring lowest unprescaled single-muon triggers~\cite{lowest_unprescaled_triggers}. 

\par An offline cut \MET~$>$ 150~\GeV is applied as the triggers are not fully efficient below that. Since the trigger turn-on curve is not well modeled in MC,
 trigger efficiencies are measured in both data and MC from a single-muon measurement region, and scale factors are calculated to correct
turn-ons in MC to match data in both the signal region and the single-muon control region.

\par The efficiencies of \MET~triggers are derived in a single-muon measurement region.
The event selection is same as the selection in the resolved regime in Chapter~\ref{ch:ana-sig}, except for the cut on \met.
The efficiencies are calculated inclusively in the number $b$-jets to allows for a larger statistics in the measurement region.

\par The \MET~trigger efficiency is defined by:

\begin{equation}
	\label{eq:trigeff}
	\text{efficiency} = \frac{\text{\#Events passed selection AND \MET~trigger requirement}}{\text{\#Events passed selection}}
\end{equation}

\par The efficiencies are calculated for each \MET~trigger separately within the data-taking periods in which it is used, and separately for data and MC.
 The trigger efficiency curves for each \MET~trigger are shown in Fig.~\ref{fig:TrigEff} as a function of \METnomu~to mimic the \MET~topology on trigger level.

\begin{figure}[tb!]
	\centering
	\includegraphics[width=0.45\textwidth]{chapters/c6/figures/METTriggerCalibration/efficiecy_HLT_xe70_mht.pdf}
	\includegraphics[width=0.45\textwidth]{chapters/c6/figures/METTriggerCalibration/efficiecy_HLT_xe90_mht_L1XE50.pdf}
	\includegraphics[width=0.45\textwidth]{chapters/c6/figures/METTriggerCalibration/efficiecy_HLT_xe110_mht_L1XE50.pdf}
	\includegraphics[width=0.45\textwidth]{chapters/c6/figures/METTriggerCalibration/efficiecy_HLT_xe110_pufit_L1XE55.pdf}
	\caption{Measured trigger efficiencies as a function of offline \METnomu~in data and MC for the \MET~triggers used in 2015-2018. The plots are shown for 0,1 and 2 $b$-tagged jets together. The lower panels provide the ratio of data and MC efficiencies (the scale factor).}
	\label{fig:TrigEff}
\end{figure}

\par Scale factors (SF) are defined as the ratio of \MET~trigger efficiencies for data and MC:

\begin{equation}
	\label{eq:dataMCsf}
	\text{SF} = \frac{\text{Efficiency}^{\text{data}}_{\mu}}{\text{Efficiency}^{\text{MC}}_{\mu}}
\end{equation}

To calculate the data-driven corrections for the MC trigger turn-on curves, the scale factors are fitted for each \MET~trigger starting in the range 100~\GeV~$<$ \METnomu~$<$ 300~\GeV~in \MET~bins of 10~\GeV~using the following fit function:

\begin{eqnarray}
	\label{eq:dataMCsf_fit}
	f\left(\text{x}\right) = p_0 \cdot \left[1 + \text{erf}\left(\frac{\text{x} - p_{1}}{\sqrt{2}p_{2}}\right)\right] + p_3
\end{eqnarray}
where $x = \METnomu$.

\par The scale factors applied to the MC in the signal and one-lepton control regions are given by evaluating $f(\met)$ or $f(\METnomu)$, respectively. 
The scale factors are shown in Fig.~\ref{fig:TrigSF} together with the fitted SF curves.
Good agreement is observed in data and MC efficiencies Figure~\ref{fig:TrigSF_validation}, after the scale factors are applied to the simulation.

\begin{figure}[tb!]
	\centering
	\includegraphics[width=0.45\textwidth]{chapters/c6/figures/METTriggerCalibration/SF_HLT_xe70_mht.pdf}
	\includegraphics[width=0.45\textwidth]{chapters/c6/figures/METTriggerCalibration/SF_HLT_xe90_mht_L1XE50.pdf}
	\includegraphics[width=0.45\textwidth]{chapters/c6/figures/METTriggerCalibration/SF_HLT_xe110_mht_L1XE50.pdf}
	\includegraphics[width=0.45\textwidth]{chapters/c6/figures/METTriggerCalibration/SF_HLT_xe110_pufit_L1XE55.pdf}
	\caption{Measured scale factors as a function of offline \METnomu~for the \MET~triggers used in 2015-2010. The scale factors are derived for 0,1 and 2 $b$-tagged jets together. The band shows the 1$\sigma$ fit uncertainty.}
	\label{fig:TrigSF}
\end{figure}

\begin{figure}[tb!]
	\centering
	\includegraphics[width=0.45\textwidth]{chapters/c6/figures/METTriggerCalibration/validation_HLT_xe70_mht.pdf}
	\includegraphics[width=0.45\textwidth]{chapters/c6/figures/METTriggerCalibration/validation_HLT_xe90_mht_L1XE50.pdf}
	\includegraphics[width=0.45\textwidth]{chapters/c6/figures/METTriggerCalibration/validation_HLT_xe110_mht_L1XE50.pdf}
	\includegraphics[width=0.45\textwidth]{chapters/c6/figures/METTriggerCalibration/validation_HLT_xe110_pufit_L1XE55.pdf}
	\caption{Validation plots showing \MET~trigger efficiencies and scale factors as function of offline \METnomu~after applying scale factor corrections for $\METnomu >$ 100~\GeV~to the MC in the full single lepton region. Good agreement is observed between data and MC.}
	\label{fig:TrigSF_validation}
\end{figure}

\par The triggers are listed in Table~\ref{tab:summary_triggers_used}. 

\begin{table}
	\scriptsize
	\begin{center}
	    \resizebox{1.\textwidth}{!}{
			\begin{tabular}{c c c c}
				\hline
			    \hline
			    Period & 0 lepton & 1 lepton & 2 lepton + \met~trigger SF measurement \\
			    \hline
			    2015 & \textsc{HLT\_xe70\_mht} & \textsc{HLT\_xe70\_mht} & \textsc{HLT\_e24\_lhmedium\_L1EM20VH}\\
			    & & & \textbf{OR} \textsc{HLT\_e120\_lhloose}  \\ 
			    & & & \textbf{OR} \textsc{HLT\_mu20\_iloose\_L1MU15} \\
			    & & & \textbf{OR} \textsc{HLT\_mu50} \\
			    \hline
			    2016 & \textsc{HLT\_xe90\_mht\_L1XE50} & \textsc{HLT\_xe90\_mht\_L1XE50} & \textsc{HLT\_e60\_lhmedium\_nod0}  \\
			    (A) & & & \textbf{OR} \textsc{HLT\_e140\_lhloose\_nod0} \\ 
			    & & & \textbf{OR} \textsc{HLT\_mu40} \\
			    & & & \textbf{OR} \textsc{HLT\_mu50} \\
			    \hline
			    2016 & \textsc{HLT\_xe90\_mht\_L1XE50} & \textsc{HLT\_xe90\_mht\_L1XE50} & \\
                (B-D3) & & & \textsc{HLT\_e60\_lhmedium\_nod0} \\
			    & & & \textbf{OR} \textsc{HLT\_e140\_lhloose\_nod0} \\ 
			    & & & \textbf{OR} \textsc{HLT\_mu24\_ivarmedium} \\
			    & & & \textbf{OR} \textsc{HLT\_mu50} \\
			    \hline
			    2016 & \textsc{HLT\_xe110\_mht\_L1XE50} & \textsc{HLT\_xe110\_mht\_L1XE50} & \textsc{HLT\_e26\_lhtight\_nod0\_ivarloose} \\	
			    (D4-E3)& & & \textbf{OR} \textsc{HLT\_e60\_lhmedium\_nod0} \\
			    & & & \textbf{OR} \textsc{HLT\_e140\_lhloose\_nod0} \\ 
			    & & & \textbf{OR} \textsc{HLT\_mu24\_ivarmedium} \\
			    & & & \textbf{OR} \textsc{HLT\_mu26\_ivarmedium} \\
			    & & & \textbf{OR} \textsc{HLT\_mu50} \\
		        \hline
			    2016 & \textsc{HLT\_xe110\_mht\_L1XE50} & \textsc{HLT\_xe110\_mht\_L1XE50} & \textsc{HLT\_e26\_lhtight\_nod0\_ivarloose}\\
                (F1)&  & & \textbf{OR} \textsc{HLT\_e60\_lhmedium\_nod0} \\
			    & & & \textbf{OR} \textsc{HLT\_e140\_lhloose\_nod0} \\ 
			    & & & \textbf{OR} \textsc{HLT\_mu26\_ivarmedium} \\
			    & & & \textbf{OR} \textsc{HLT\_mu50} \\
			    \hline
			    2016 & \textsc{HLT\_xe110\_mht\_L1XE50} & \textsc{HLT\_xe110\_mht\_L1XE50} &\textsc{HLT\_e26\_lhtight\_nod0\_ivarloose} \\
			    (F2-) & & & \textbf{OR} \textsc{HLT\_e60\_lhmedium\_nod0} \\
			    & & & \textbf{OR} \textsc{HLT\_e140\_lhloose\_nod0} \\ 
			    & & & \textbf{OR} \textsc{HLT\_mu26\_ivarmedium} \\
			    & & & \textbf{OR} \textsc{HLT\_mu50} \\
			    \hline
			    2017 & \textsc{HLT\_xe110\_pufit\_L1XE55} & \textsc{HLT\_xe110\_pufit\_L1XE55} &\textsc{HLT\_e60\_lhmedium\_nod0} \\
			    & & & \textbf{OR} \textsc{HLT\_e140\_lhloose\_nod0} \\
			    & & & \textbf{OR} \textsc{HLT\_mu26\_ivarmedium} \\
			    & & & \textbf{OR} \textsc{HLT\_mu50} \\
			    \hline
			    2018 & \textsc{HLT\_xe110\_pufit\_70\_L1XE55} & \textsc{HLT\_xe110\_pufit\_70\_L1XE55} &  \textsc{HLT\_e60\_lhmedium\_nod0} \\
			    (B-C5)& & & \textbf{OR} \textsc{HLT\_e140\_lhloose\_nod0} \\
			    & & & \textbf{OR} \textsc{HLT\_mu26\_ivarmedium} \\
			    & & & \textbf{OR} \textsc{HLT\_mu50} \\
			    \hline
			    2018 & \textsc{HLT\_xe110\_pufit\_65\_L1XE55} & \textsc{HLT\_xe110\_pufit\_65\_L1XE55} &\textsc{HLT\_e60\_lhmedium\_nod0} \\
			    (C5-) & & & \textbf{OR} \textsc{HLT\_e140\_lhloose\_nod0} \\
			    & & & \textbf{OR} \textsc{HLT\_mu26\_ivarmedium} \\
			    & & & \textbf{OR} \textsc{HLT\_mu50} \\    		
			    \hline
			    \hline
		    \end{tabular}
		}
	\end{center}
	\caption{\met~and single-lepton triggers used in the analysis.}
	\label{tab:summary_triggers_used}
\end{table}
