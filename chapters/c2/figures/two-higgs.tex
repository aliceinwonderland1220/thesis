\documentclass[border=1pt]{standalone}
\usepackage{tikz}
\usepackage[compat=1.0.0]{tikz-feynman}
\usepackage{contour}

\begin{document}
\begin{tikzpicture}
  \begin{feynman}
            \vertex (a);
          \vertex [above left=of a](p1);
          \vertex [below left=of a](p2);
          \vertex [above left=of p1](q1){g};
          \vertex [below left=of p2](q2){g};
          \vertex [right=of a] (b);
          \vertex [above right=of b] (d){\(h\)};
          \vertex [below right=of b] (c);
          \vertex [above right=.5cm of c] (f3) {\(\bar{\chi}\)};
          \vertex [below right=.5cm of c] (f2) {\(\chi\)};
      %    \vertex[label={left: $g_q$}] (s1) at (-.6,0);
      %    \vertex[label={right: $g_q$}] (s2) at (.6, 0);
   %     \vertex (a) at (-2, 1) {\(q\) };
   %     \vertex (b) at (-2,-1) {\(\bar{q}\) };
   %     \vertex (c) at ( 2,1) {\(q\)};
   %     \vertex (d) at ( 2,-1) {\(\bar{q}\)};

      \diagram [layered layout, horizontal = a to b] {
          (q1)  -- [gluon] (p1) -- [,edge label'=\(t\)] (p2) -- [gluon] (q2),
      (p1) -- [plain,  edge label'=\(t\)] (a) -- [plain, edge label'=\(t\)] (p2),
      (a)--[scalar, edge label'=\(A\)](b)--[scalar](d),
      (b)--[scalar, edge label'=\(a\)](c),
      (f2)--[fermion](c)--[fermion](f3),

 % e -- [gluon]  f,
 % h -- [fermion] f -- [fermion] i;
};
  \end{feynman}
\end{tikzpicture}
\end{document}
