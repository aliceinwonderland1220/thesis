
\chapter{The Standard Model}


\label{ch:sm}


\section{Introduction}
\par What are the most fundamental particles that constitute our world? How do the particles interact with each other?
These are two of the essential questions.


\par By the early twentieth century, scientists believed that atoms were the building blocks of nature. People trusted Newtonian laws of motion which solved most problems of physics.
Fundamental precepts of physics at that time were challenged by the establishment of Einstein's theory of relativity and quantum mechanics.
Just as Max Born predicted, ``Physics as we know it will be over in six months''.


\par Guided by both theoretical models and experimental discoveries, quantum field theory became very successful in describing particles as well as electromagnetic and weak interactions between them. The first evidence for a subatomic particle, proton, was found in 1919 and followed by the emergence of the strong force in 1921 as a new interaction type that holds the subatomic particles. 


\par Starting from 1964 when Murray Gell-Mann and George Zweig put forth the idea of quarks, 
the Standard Model was gradually completed and finally summarized by John Iliopoulos in 1974.


\par In Sections 2.2 and 2.3, an overview of the Standard Model of particle physics and its interactions are given, while
a brief description of gauge theory follows in Section 2.4. We introduce the Standard Model Lagrangian in Section 2.5 to explain how particles gain their masses via the Higgs mechanism in Section 2.6.


\section{Particles}
% Standard Model particle table
\begin{figure}[htbp]
  \centering
  \includegraphics[width=0.8\textwidth]{chapters/c1/figures/SM-particle-table}
  \caption{Particles of the Standard Model of particle physics}
  \label{fig:c1Standard Modelparticletable}
\end{figure}
\par Shown in Fig.~\ref{fig:c1Standard Modelparticletable}, the fundamental building blocks of matter are fermions with $\frac{1}{2}$ 
spin in the Standard Model, while the mediators of forces of their interactions are gauge bosons with one spin. 
The spin-0 scalar Higgs boson is expected to be at the origin of the mass of other particles.


\begin{itemize}
 \item \textbf{Fermions} can be divided into two categories: quarks, and leptons. There are three generations of both kinds: the first generation made up the common matter, and the higher generations can be accessed at higher energies.
\begin{itemize}
 \item \textbf{Quarks} Each generation of quarks has two types: up-type with a charge of $\frac{2}{3}$ and down-type with a charge of -$\frac{1}{3}$ and both
   can interact through the electromagnetic, strong and weak interactions. The color charge carried by quarks makes them able to interact via the strong interactions. Asymptotic confinement explains why quarks need to bind together at small distances, resulting in the color-neutral particles called \textit{hadrons}. Quarks appear to be free because of the decreasing strong coupling strength within short distances, hence the quark jets at high energy are initially single free quarks flying
out from their point of creation.
    There are two types of hadrons: \textit{baryons} and \textit{mesons}. Protons and neutrons are two examples of baryons while pions and kaons are examples of mesons. Weak isospin makes quarks able to the couple via weak interaction like fermions.
 \item \textbf{Leptons} Each generation has a charged particle and an electrically neutral neutrino. Leptons don't have a color charge so they can't interact via strong interactions. While the charged particles can interact through the electromagnetic and weak forces as they have both charge and weak isospin, neutrinos only interact via weak interactions, making them extremely hard to detect in experiments.
\end{itemize}
 \item \textbf{Bosons} as mediators for the three types of force in the Standard Model are listed below.
\begin{itemize}
 \item \textbf{Photons}: discovered very early, it is the mediator of the electromagnetic force. It's a massless, electromagnetic-charge-neutral, spin-1 particle.
 \item \textbf{Gluons}: first discovered at DESY in the late 1970s, it is the mediator of the strong force. 
   Gluons are massless, spin-1 particles. The gluon carries color charge itself and interacts with quarks. 
    There are eight varieties of gluons as there are nine different combinations of the color charge, but the singlet state $\frac{r\bar{r}+b\bar{b}+g\bar{g}}{\sqrt{3}}$ does not exist.


 \item The $W^{\pm}$ and Z bosons: discovered in the late 1980s, serve as mediators of the weak force. They are spin-1 particles with masses around 80 and 91~\GeV. The \textbf{$W^{\pm}$} bosons carry the weak charged current and have electromagnetic charge of $\pm 1$, while the Z boson is the mediator of the weak neutral current and is electromagnetic-charge-neutral. 
  \item \textbf{Higgs} boson, discovered by the CMS and ATLAS collaborations in 2012~\cite{Aad:2012tfa,Chatrchyan:2012xdj}, is a spin-0 scalar neutral particle with mass 125~\GeV. Fermions and bosons acquire their mass through Yukawa coupling with the Higgs field. 
\end{itemize}
\end{itemize}
 
 


\section{Interactions}
\begin{table}[tbh]
\centering
\tiny
\begin{tabular}{|l|c|c|c|c|c|c|c}


\hline
  Force typle & Mediator & Affected particles & Acts on & Coupling\\
  &&&&Constant\\ 
  \hline
\hline
  Electromagnetic & $\gamma$&Electrically charged fermions&Electric charge& $\alpha$ \\
  Strong & $g$ &Quarks & Color charge& $\alpha_s$ \\
  Weak & $W^{\pm}, Z$ &Left-handed fermions & Weak hypercharge& $\alpha_W$ \\
\hline
\end{tabular}
\caption{Three fundamental interactions in Standard Model. }
\label{tab:forces}
\end{table}


\par Table~\ref{tab: forces} summarizes the properties of the three different interactions, including the values of coupling constants at low energy. 
At low energies, the weak force is much weaker than the electromagnetic force, even though the weak coupling constant is relatively larger. 
This is because the strength of weak interactions is suppressed by the large masses of the W and Z bosons. 
%The weak force becomes stronger than the electromagnetic force as values of coupling constants change.


\section{Gauge theory}
\par Gauge Theories are field theories in which the Lagrangian is invariant under certain Lie groups of local transformations.
\par As a gauge theory, the Standard Model Lagrangian is invariant under transformations of the group $SU(3)_c \times SU(2)_L \times U(1)_Y$. The fermions of the Standard Model are described by representations of the symmetry group, while the local gauge symmetry is represented by a force mediated by gauge bosons.


\par There are twelve gauge bosons in total: eight gluons which correspond to the generators of $SU(3)_c$, $W^{\pm}$ bosons which correspond to generators of $SU(2)_L$, and the Z boson and $\gamma$ which correspond to linear combinations of generators for $SU(2)_L \times U(1)_Y$.


\par For fermions, two different representations are used based on chirality: left-handed fermions are doublets under SU(2) and interact with the weak bosons, while right-handed fermions are singlets.
\begin{equation}
 \psi_L^{j}=\left( \begin{smallmatrix} \psi_{L+}^{j}\\ \psi_{L-}^{j} \end{smallmatrix}\right), \psi_{R\sigma}^{j}
 \label{eq:fermion}
\end{equation}
Here $j=1,2,3$ is the generation index. For quarks, $\sigma=+$ represents up-type quark and $\sigma=-$ represents down-type quarks, while for leptons, $\sigma=+$ represents neutrinos and $\sigma=-$ represents charged leptons. As neutrinos are considered to be massless, there are no right-handed neutrinos.


%The Standard Model Lagrangian is shown in Eq~\ref{eq:c1Standard Modell}:
\section{The formation of the Lagrangian of the Standard Model}
% Standard Model equation
\par To understand why the Higgs field is responsible for the masses of other particles, the Lagrangian of the Standard Model is examined.


\begin{equation}
  \mathcal{L}_{Standard\ Model}= \mathcal{L}_{Gauge}+ \mathcal{L}_{Fermion}+ \mathcal{L}_{Higgs}+ \mathcal{L}_{Yukawa},
 \label{eq:Standard Modelall}
\end{equation}
where $\mathcal{L}_{Gauge}$ describes the kinematics of the gauge fields, which are written as
\begin{equation}
  \mathcal{L}_{Gauge}= -\frac{1}{4}G_{a\mu\nu}G^{\mu\nu}_a - \frac{1}{4}W_{a\mu\nu}W^{\mu\nu}_a - \frac{1}{4}B_{\mu\nu}B^{\mu\nu}.
\end{equation}
Here $W_{\mu\nu}^a$ and $B_{\mu\nu}$ are the field tensors corresponding to non-Abelian SU(2) and Abelian U(1) respectively.
$\mathcal{L}_{Fermion}$ describes the fermion kinematics and interactions with gauge bosons and is written as:
\begin{equation}
  \mathcal{L}_{Fermion}=\sum_j \overline{\psi^j_L}i\gamma^\mu D^L_\mu \psi^j_L + \sum_{j, \sigma} \overline{\psi^j_{R\sigma}}i\gamma^{\mu}D^R_{\mu}\psi^j_{R\sigma},
  \label{eq:smf}
\end{equation}
where $\gamma^{\mu}$ are the Dirac matrices and $D_{\mu}$ is the covariant derivative operator which is defined as 
\begin{equation}
  D_\mu=\partial_\mu-\frac{ig_1Y}{2}B_\mu-\frac{ig_2\tau^i}{2}\mathbf{W}^i_\mu-i\frac{ig_3\lambda^a}{2}\mathbf{G}^a_\mu.
  \label{eq:dirac}
\end{equation}
Here Y, $\tau^i$, and $\lambda$ are the generators for the U(1), SU(2) and SU(3) gauge symmetry groups, and $g_1$, $g_2$ and $g_3$ 
are coupling constants between fermion and gauge fields. 
$B_\mu$ is the spin-one field needed to maintain the U(1) gauge invariance,
and $\mathbf{W}_\mu$ and $\mathbf{G}_\mu$ are composed of $2\times3$ and $3\times3$ traceless Hermitian matrices.
They are associated with the field tensors above via:


\begin{equation*}
 B_{\mu\nu}=\partial_\mu B_\nu-\partial_\nu B_\mu,
\end{equation*}


\begin{equation*}
 \mathbf{W}_{\mu\nu}=\partial_\mu\mathbf{W}_\nu-\partial_\nu\mathbf{W}_\mu+ig_2\frac{\left(\mathbf{W}_\mu\mathbf{W}_\nu-\mathbf{W}_\nu\mathbf{W}_\mu\right)}{2}, and 
\end{equation*}


\begin{equation*}
 \mathbf{G}_{\mu\nu}=\partial_\mu\mathbf{G}_\nu-\partial_\nu\mathbf{G}_\mu+\left(\mathbf{G}_\mu\mathbf{G}_\nu-\mathbf{G}_\nu\mathbf{G}_\mu\right).
\end{equation*}
The gauge bosons, the charged $W^{\pm}$, photon $\gamma$ ($A_\mu$) and $Z_\mu$ bosons arise as follows:
\begin{equation*}
 A_\mu=W_{11\mu}\sin\left(\theta_w\right)+B_\mu\cos\left(\theta_w\right),
\end{equation*}


\begin{equation*}
 Z_\mu=W_{11\mu}\cos\left(\theta_w\right)-B_\mu\sin\left(\theta_w\right),
\end{equation*}


\begin{equation*}
 W_\mu^+=W_\mu^{-*}=\frac{W_{12\mu}}{\sqrt{2}}.
\end{equation*}


\par Higgs term describes the kinematic and potential energies of the Higgs field $\phi$:
\begin{equation}
 \mathcal{L}_{Higgs}= T-V =\overline{\left(D_\mu\phi\right)}D^\mu\phi-\mu^2\bar{\phi}\phi-\lambda(\bar{\phi}\phi)^2
 \label{eq:higgs}
\end{equation}


Finally, the Yukawa term describes the interactions between matter particles and the Higgs field is given by
\begin{equation}
 \mathcal{L}_{Yukawa} = - g_l \overline{L_l}\phi l_R - g_d\overline{Q_L}\phi d_R - g_u\overline{Q_L}\phi_c u_R + (h.c.)
 \label{eq:yukawa}
\end{equation}
\par Fermion mass terms $m(\overline{\psi_R}\psi_L+\overline{\psi_L}\psi_R)$ and gauge boson mass term s$\frac{1}{2}m^2 B^\mu B_\mu$ would break the SU(2) invariance of the Lagrangian. Thus the Higgs mechanism is introduced to explain mass generation through the electroweak symmetry breaking mechanism.


\section{Spontaneous Symmetry Breaking and the Higgs mechanism}
\par The Higgs field $\phi$ is introduced to break the electroweak symmetry in a vacuum (spontaneous symmetry breaking).
As a doublet in SU(2), we denote $\phi$ as
\begin{equation}
 \phi_0=\frac{1}{\sqrt{2}}\left( \begin{smallmatrix} \phi_1+i\phi_2\\ \phi_3+i\phi_4 \end{smallmatrix}\right).
 \label{eq:higgsfield}
\end{equation}


\par We need to look at $ \mathcal{L}_{Higgs}$ to understand how the gauge bosons gain masses via the Higgs mechanism Model.\\
\par For $\mu^2$<$0$, the minimum of potential V in Equation~\ref{eq:higgs} is at $\bar{\phi}\phi=-\frac{\mu^2}{2\lambda}=\frac{\nu^2}{2}$
\par Since the potential depends only on the combination $\bar{\phi}\phi$, we can arbitrarily choose the vacuum:
\begin{equation}
 \phi_0=\left( \begin{smallmatrix} 0\\v \end{smallmatrix}\right).
 %\label{eq:vac}
\end{equation}


\par We can expand Higgs field around vacuum as:
\begin{equation}
 \phi=\frac{1}{\sqrt{2}}\left( \begin{smallmatrix} 0\\v + h \end{smallmatrix}\right).
 %\label{eq:vac}
\end{equation}


\par The kinematic term $T=\overline{\left(D_\mu\phi\right)}D^\mu\phi $ gives us the mass terms of bosons:
 $\frac{1}{2}(\frac{1}{2}\nu g_2)^2 W_\nu^+ W^{-\nu}$ and $\frac{1}{2}(\frac{1}{2}\nu \sqrt{g_1^2+g_2^2})^2 Z_\nu Z^\nu$.\\


\par By plugging the simplified Higgs doublet into the Yukawa lagrangian term in Equation~\ref{eq:yukawa}, we obtain the reduced form:
\begin{equation}
 \mathcal{L}_{Yukawa} = - \sum_{fermions} m_f \overline{\psi_f}\psi_f - \sum_{fermions} \frac{m_f}{v} \overline{\psi_f}\psi_f h,
 \label{eq:vac}
\end{equation}
where $m_f=\frac{1}{\sqrt{2}} g_f v $ yields the mass of the fermions and the second term represents the interaction between fermions and the Higgs boson with the interaction Yukawa coupling proportional to the fermion mass.


%\begin{equation}
% \begin{alignedat}{2}
% L = & -\frac{1}{4}B_{\mu\nu}B^{\mu\nu} - \frac{1}{8}tr(F_{\mu\nu}F^{\mu\nu}) - \frac{1}{2}tr(G_{\mu\nu}G^{\mu\nu}), (Gauge \, terms) \\
%   & +\begin{pmatrix} \bar{\nu}_{L} & \bar{e}_{L} \end{pmatrix}\bar{\sigma}^{\mu}iD_{\mu}\begin{pmatrix} \nu_{L} \\ e_{L} \end{pmatrix} + \bar{e}_{R}\sigma^{\mu}iD_{\mu}e_{R} + \bar{\nu}_{R}\sigma^{\mu}iD_{\mu}\nu_{R}, (Lepton \, dynamical \, terms) \\
%   & -\frac{\sqrt{2}}{\upsilon}[\begin{pmatrix} \bar{\nu}_{L} & \bar{e}_{L} \end{pmatrix}\phi M^{e}e_{R} + \bar{e}_{R}\bar{M}^{e}\bar{\phi}\begin{pmatrix} \nu_{L} \\ e_{L} \end{pmatrix}], (Electron, muon, Tau \, mass \, terms) \\
%   & -\frac{\sqrt{2}}{\upsilon}[\begin{pmatrix} -\bar{e}_{L} & \bar{\nu}_{L} \end{pmatrix}\phi^{*} M^{\nu}\nu_{R} + \bar{\nu}_{R}\bar{M}^{\nu}\phi^{T}\begin{pmatrix} -e_{L} \\ \nu_{L} \end{pmatrix}], (Neutrino \, mass \, terms) \\
%   & +\begin{pmatrix} \bar{u}_{L} & \bar{d}_{L} \end{pmatrix}\bar{\sigma}^{\mu}iD_{\mu}\begin{pmatrix} u_{L} \\ d_{L} \end{pmatrix} + \bar{u}_{R}\sigma^{\mu}iD_{\mu}u_{R} + \bar{d}_{R}\sigma^{\mu}iD_{\mu}d_{R}, (quark \, dynamical \, terms) \\
%   & -\frac{\sqrt{2}}{\upsilon}[\begin{pmatrix} \bar{u}_{L} & \bar{d}_{L} \end{pmatrix}\phi M^{d}d_{R} + \bar{d}_{R}\bar{M}^{d}\bar{\phi}\begin{pmatrix} u_{L} \\ d_{L} \end{pmatrix}], (Down, strange, bottom \, mass \, terms) \\
%   & -\frac{\sqrt{2}}{\upsilon}[\begin{pmatrix} -\bar{d}_{L} & \bar{u}_{L} \end{pmatrix}\phi^{*} M^{u}u_{R} + \bar{u}_{R}\bar{M}^{u}\phi^{T}\begin{pmatrix} -d_{L} \\ u_{L} \end{pmatrix}], (Up, charm, top \, mass \, terms) \\
%   & +\bar{D_{\mu}\phi}D^{\mu}\phi - m_{h}^{2}[\bar{\phi}\phi-\upsilon^{2}/2]^{2}/2\upsilon^{2}, (Higgs \, dynamical \, and \, mass \, terms)
% \label{eq:c1Standard Modell}
% \end{alignedat}
%\end{equation}
%
%The definition of derivative operators in the Eq~\ref{eq:c1Standard Modell} are:
%\begin{equation}
% \begin{aligned}
% D_{\mu}\begin{pmatrix} \nu_{L} \\ e_{L} \end{pmatrix} = [\partial_{\mu}-\frac{ig_{1}}{2}B_{\mu}+\frac{ig_{2}}{2}W_{\mu}]\begin{pmatrix} \nu_{L} \\ e_{L} \end{pmatrix} \\
% D_{\mu}\nu_{R} = \partial_{\mu}\nu_{R},\quad D_{\mu}e_{R} = [\partial_{\mu}-ig_{1}B_{\mu}]e_{R}
% \end{aligned}
% \label{eq:c1Standard Modelldl}
%\end{equation}
%
%\begin{equation}
% \begin{aligned}
% D_{\mu}\begin{pmatrix} u_{L} \\ d_{L} \end{pmatrix} = [\partial_{\mu}+\frac{ig_{1}}{6}B_{\mu}+\frac{ig_{2}}{2}W_{\mu}+igG_{\mu}]\begin{pmatrix} u_{L} \\ d_{L} \end{pmatrix} \\
% D_{\mu}u_{R} = [\partial_{\mu}+\frac{i2g_{1}}{3}B_{\mu}+igG_{\mu}]u_{R},\quad D_{\mu}d_{R} = [\partial_{\mu}-\frac{ig_{1}}{3}B_{\mu}+igG_{\mu}]d_{R}
% \end{aligned}
% \label{eq:c1Standard Modelldq}
%\end{equation}
%
%\begin{equation}
% \begin{aligned}
% D_{\mu}\phi = [\partial_{\mu}+\frac{ig_{1}}{2}B_{\mu}+\frac{ig_{2}}{2}W_{\mu}]\phi
% \end{aligned}
% \label{eq:c1Standard Modelldh}
%\end{equation}


\section{Challenges}
\par Despite being remarkably successful, the Standard Model has its limitations and leaves us with some significant questions.


\par First of all, while the Higgs boson gives mass to other particles via couplings, the Standard Model gives no prediction for
the Higgs boson mass. The current measurement of the Higgs mass indicates that the electroweak scale is $\mathcal{O}$(100~\GeV), 
while the Planck scale is at $\mathcal{O}$($10^{19}$~\GeV). This so-called hierarchy problem indicates that either unnatural fine-tuning exists, or some new physics cancels out the divergent terms in the Higgs boson mass.


%\par Second, the Standard Model does not incorporate gravity. There is the grand unification which unifies the three gauge interactions in the Standard Model, electromagnetic, weak, and strong interactions. But what about gravity? Why gravity is so weak compared to the other three interactions? How does gravity might merge into a greater symmetry?


%\par Another challenge is to understand neutrino oscillations, which indicates that neutrino has mass and its flavor might oscillate under the PMNS matrix as quark oscillates under the CKM matrix, in contradiction with the prediction of the Standard Model.


\par The source of the huge imbalance between matter and antimatter (baryon asymmetry) in the universe is unexplained. 
It's reasonable to assume that equal amounts of matter and antimatter are created during the Big Bang. However, 
the universe is now dominated by matter while antimatter has essentially vanished. This requires Charge Parity(CP) violation.
CP violation in the Standard Model is insufficient to account for the level of matter-antimatter asymmetry observed currently.


\par Finally, one striking evidence of physics beyond the Standard Model is dark matter. Given astrophysics theory and observations, 
22.7\% of the total mass-energy of the universe is dark matter, which is about five times more than visible matter. 
Although both direct and indirect evidence proves the existence of dark matter, the dark matter remains a mystery. 
So far, the properties of dark matter have only been probed via its gravitational interaction. The leading hypothesis suggests that 
most of the dark matter is in the form of stable, electrically neutral, massive particles, i.e. Weakly Interacting Massive Particle(WIMB).


\par To answer these questions, several well-motivated models predict dark matter interacting with Standard Model particles weakly, perhaps via a new mediator. If this is the case, then there is a good reason to search for dark matter production in high energy collisions, such as those provided by the Large Hadron Collider.