\chapter{The LHC}
\label{ch:lhc}
%P1, what is the LHC?
\par The large hadron collider, abbreviated as LHC, is the most powerful proton-proton collider located at CERN, Geneva, Switzerland. The LHC is installed in a 26.7 km tunnel that was built for last generation lepton collider between 1984 and 1989. The LHC tunnel, which is located from 45m to 170m below the surface, is composed of 8 straight sections and 8 arcs. Therefore, the LHC can be viewed as an octant~\ref{fig:xxx}. Every division of the octant, also known as “Point”, has an access elevator from surface to underground. Half of the LHC points are hosing the detector systems currently: ATLAS\cite{Aad:2008zzm} at Point 1, ALICE\cite{Aamodt:2008zz} at Point 2, CMS\cite{Chatrchyan:2008aa} at Point 5 and LHCb\cite{Alves:2008zz} at Point 8. The other 4 points are designed for the LHC operation purposes. 

%P2, the LHC, history and future
\par Rome wasn’t built in one day, so does the LHC. The LHC was built on the infrastructure of the previous generation of colliders that located at CERN. The LHC is the current frontier of the evolution chain: from Proton Synchrotron (1954)\cite{Gilardoni:2011za}, Super Proton Synchrotron (1976)\cite{Doble:2017syb}, Large Electron-Positron Collider (1984)\cite{LepInjectorStudy:1983aa}\cite{LepInjectorStudy:1983ab}, to Large Hadron Collider (2008)\cite{Bruning:2004ej}\cite{Buning:2004wk}. Both infrastructure and technology are utilized in the economic manner to support the most powerful collider in the world, the LHC. The proton beam can be accelerated at 7 TeV by the current LHC setup. Although the LHC has reached the frontier of the high energy of human experiment, it is not the end of the CERN collider evolution chain. Both the LHC upgrade plan\cite{ApollinariG.:2017ojx} and next generation collider design\cite{Benedikt:2018csr} have been proposed to sustain the prosperity of the CERN collider family.

\section{The LHC overview}
\label{sec:lhcs1}

%P1, LHC, ring, size, location
\par Sample text sample text sample text. Sample text sample text sample text.

%P2, LHC, 8 straight section and 8 arcs
\par Sample text sample text sample text. Sample text sample text sample text.

\section{The LHC performance}
\label{sec:lhcs2}
%P1, the LHC performance and collision data collection: how many data do we take at what energy?
\par Sample text sample text sample text. Sample text sample text sample text.

\subsection{Luminosity}
%P1, how many data, lumi, definition, equation
\par Sample text sample text sample text$N=L\sigma$. Sample text sample text sample equation~\ref{eq:c3lumi}, where $N_{b}$ is the number of protons per bunch, $n_{b}$ is number of filled bunches per beam, $f_{r}$ is the frequency of the beam circling the ring, $\gamma_{r}$ relativistic gamma factor of the protons, $\epsilon_{n}$ is the normalized transverse beam emittance, $\beta^{*}$ is the measure of beam width in longitude direction, $F$ is a geometric factor which accounts for the non-zero crossing angle between two beams.

\begin{equation}
  L = \frac{N_{b}^{2}n_{b}f_{r}\gamma_{r}4\pi\epsilon_{n}\beta^{*}}{F}
  \label{eq:c3lumi}
\end{equation}

%P2, how to determine lumi? Do we need this?
\par Sample text sample text sample text. Sample text sample text sample text.

\subsection{Beam energy and Pile-up}
%P1, what energy? energy to pile-up, physica insensitive, but detector does sensitive, trigger
\par Sample text sample text sample text. Sample text sample text sample text.

%P2, pile-up definition, and suppression
\par Sample text sample text sample text. Sample text sample text sample text.

\section{The LHC operation}
\label{sec:lhcs3}
%P1, the LHC operation and detector operation
\par Sample text sample text sample text. Sample text sample text sample text.

\subsection{Machine accelerator}
\par Sample text sample text sample text. Sample text sample text sample text.

\subsection{Machine beam}
\par Sample text sample text sample text. Sample text sample text sample text.
