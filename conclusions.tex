\chapter{Conclusions}

\label{ch:con}

\par A search for dark matter coupled to the Higgs boson has been presented in this thesis. 
The data sample, collected in 2015, 2016, 2017 and 2018 with the ATLAS detector at the CERN Large Hadron Collider (LHC), corresponds to an integrated luminosity of 139~\ifb. 

\par The Standard Model backgrounds are estimated using Monte-Carlo simulation and controlled by control regions. 
The analysis improves upon previous studies by adopting a new jet definition, the variable-radius track jets, to identify b-jets from highly-boosted Higgs bosons on the high mass region. 
And the improvement in low resonance masses region mainly comes from a factor of 1.75 increase in integrated luminosity. No excess of events above the expected Standard Model background is observed. 
The analysis result is interpreted in the context of a simplified Z'+2HDM model as 95\% confidence level upper limits on the signal production cross-sections, and then converted into mass limits. 
Model parameters are constrained with newly collected data, which is good guidance for further theoretical development.
