\chapter{Introduction}
Over the past few decades, the Standard Model (SM) has been established to describe the fundamental building blocks of the universe and their interactions.
 Not only the SM is elegant and self-consistent, but also its predictions agreed with particle physics experiments across a wide range of energy scales at precision levels. 


As extend our particle colliders to the corresponding energy scale, Scientists observed all force carriers, the gauge bosons, predicted by the SM.  
The Higgs boson, which plays a key role in the SM through spontaneous symmetry breaking was discovered in 2012 by both ATLAS and CMS experiment at a confidence level beyond 5$\sigma$,
  and completes the SM. \cite{Aad:2012tfa} \cite{Chatrchyan:2012xdj}


Despite being remarkably successful, the SM has its limitations and left us with a lot  of opening questions. 
New hypothetical models are established, from supersymmetry(SUSY) to extra dimensions,  
in order to break down the huge iceberg in front of us. Even though, none of the theories are proved by experiments up till now, 
people believe that the increase of collision energy and luminosity in LHC will benefit the search for new physics.
In this thesis, we will test two theoretical dark matter models, Z'+2HDM and 
2HDM+a. 

In order to probe new physics at an energy scale higher than electroweak scale,
 we trace the particles decayed from proton-proton collisions using A Toroidal LHC ApparatuS(ATLAS) detector,
 which is one of the major detectors in European Organization for Nuclear Research(LHC). This work presents a search for Dark Matter particles associated 
with the Higgs Boson decaying into a $b\bar{b}$ quark pair, using 160 fb$ ^{-1}$ of proton-proton collision data collected at a center-of-mass energy of 13 TeV 
by the ATLAS detector at the Large Hadron Collider during Run 2.

Specially, we look at the channel where the Higgs boson decays to a pair of bottom quarks, taking advantage of the large branching ratio of the $b\bar{b}$ mode. 
Two methods of Higgs boson reconstruction are used, the resolved channel where the Higgs boson is reconstructed as two separate b-quark jets, and the other boosted channel where the Higgs boson is reconstructed as a single large-radius jet using jet substructure techniques. 

 This thesis is structured as follows. 
In chapter \ref{ch:sm}, the standard model of particle physics and its challenges are briefly described. 
The dark matter and two Higgs doublet theory is mentioned as a candidate theory to resolve these challenges in chapter \ref{ch:dm}. 
Then, the CERN Large Hadron Collider and ATLAS detector are mentioned in detail as an experimental solution to examine the 
generic new particle physics theories in chapter \ref{ch:lhc} and chapter \ref{ch:atlas}. 
The physics analytic workflow and statistical interpretation are demonstrated from chapter 6 to 10. 
Finally, the conclusion is shown in chapter \ref{ch:res}.

