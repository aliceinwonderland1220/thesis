\chapter{Introduction}

\label{ch:intro}
%P1, particle physics in general -> the Standard Model
\par Over the past few decades, the Standard Model has been established to describe the fundamental building blocks of the universe and their interactions.
 Not only is the Standard Model elegant and self-consistent, but also scientists observed all force carriers, predicted by the Standard Model with the evolution of particle colliders to the corresponding energy scale.
The Higgs boson, which plays a key role in the Standard Model through spontaneous symmetry breaking was discovered in 2012 by both ATLAS and CMS collaborations, 
and completes the Standard Model\cite{Aad:2012tfa} \cite{Chatrchyan:2012xdj}.

%P3, the Standard Model of particle physics -> flaw -> dark matter
\par Despite being remarkably successful, the Standard Model has limitations and leaves us with some questions. 
New hypothetical models have been developed, like supersymmetry (SUSY) and extra dimensions,
to break down the huge iceberg in front of us. The increase of collision energy and luminosity at the large hadron collider will benefit the search for new physics.

%P4, dark matter -> search for dark matter in the particle physics way -> the LHC, ATLAS
\par In order to probe new physics at an energy scale higher than electroweak scale,
 we measure the particles produced in proton-proton collisions using A Toroidal LHC ApparatuS (ATLAS) detector system,
 which is one of the two general purpose detectors at the CERN LHC. This work presents a search for Dark Matter particles produce in association
with the Higgs Boson, using 160 $fb^{-1}$ of proton-proton collision data collected at a center-of-mass energy of 13 TeV
by the ATLAS detector at the Large Hadron Collider during Run 2. Two theoretical dark matter models, Z'+2HDM and 
2HDM+a, are used to interpret the thesis.

%P5, search method -> physics Higgs tagger, new channel, models
\par In particular, we examine the channel where the Higgs boson decays to a pair of bottom quarks, taking advantage of the large branching ratio of the $b\bar{b}$ mode. 
Two Higgs boson reconstruction methods are used, the resolved channel where the Higgs boson is reconstructed as two separate b-quark jets, 
and the boosted channel where the Higgs boson is reconstructed as a single large-radius jet.

%P6, thesis structure
\par This thesis is structured as follows. 
In Chapter~\ref{ch:sm}, the Standard Model of particle physics and its open questions are briefly described. Dark matter and two Higgs doublet models, as a candidate theory to resolve the dark matter challenges, are exposed in Chapter~\ref{ch:dm}. 
Then, the CERN Large Hadron Collider and ATLAS detector are explained in detail as an experimental solution to examine new particle physics theories in Chapter~\ref{ch:lhc} and Chapter~\ref{ch:atlas}. 
The physics analysis workflow and statistical interpretation are demonstrated from Chapter~\ref{ch:ana-intro} to~\ref{ch:res}. 
Finally, the conclusions are drawn in Chapter~\ref{ch:con}.
