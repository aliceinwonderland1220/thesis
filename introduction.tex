\chapter{Introduction}

\label{ch:intro}
%P1, particle physics in general -> the Standard Model
\par Over the past few decades, the Standard Model has been established to describe the fundamental building blocks of the universe and their interactions.
 Not only the Standard Model is elegant and self-consistent, but also scientists observed all matter particles and force carriers, predicted by the Standard Model as extending our particle colliders to the corresponding energy scale.
The Higgs boson, which plays a key role in the Standard Model through spontaneous symmetry breaking was discovered in 2012 by both ATLAS and CMS experiment, 
and completes the Standard Model\cite{Aad:2012tfa} \cite{Chatrchyan:2012xdj}.

%P3, the Standard Model of particle physics -> flaw -> dark matter
\par Despite being remarkably successful, the Standard Model has its limitations and left us with a lot of opening questions. 
New hypothetical models are established, like supersymmetry (SUSY) and extra dimensions,
to break down the huge iceberg in front of us. Even though, none of the theories are proved by experiments up till now, 
people believe that the increase of collision energy and luminosity in LHC will benefit the search for new physics.

%P4, dark matter -> search for dark matter in the particle physics way -> the LHC, ATLAS
\par In order to probe new physics at an energy scale higher than electroweak scale,
 we trace the particles decayed from proton-proton collisions using A Toroidal LHC ApparatuS (ATLAS) detector system,
 which is one of the major detectors in European Organization for Nuclear Research (CERN). This work presents a search for Dark Matter particles associated
with the Higgs Boson decaying into a $b\bar{b}$ quark pair, using 160 $fb^{-1}$ of proton-proton collision data collected at a center-of-mass energy of 13 TeV
by the ATLAS detector at the Large Hadron Collider during Run 2. Two theoretical dark matter models, Z'+2HDM and 
2HDM+a, are interpreted in this thesis.

%P5, search method -> physics Higgs tagger, new channel, models
\par Specially, we examined the channel where the Higgs boson decays to a pair of bottom quarks, taking advantage of the large branching ratio of the $b\bar{b}$ mode. 
Two Higgs boson reconstruction methods are used, the resolved channel where the Higgs boson is reconstructed as two separate b-quark jets, 
and the boosted channel where the Higgs boson is reconstructed as a single large-radius jet.

%P6, thesis structure
\par This thesis is structured as follows. 
In chapter~\ref{ch:sm}, the Standard Model of particle physics and its challenges are briefly described. The dark matter and two Higgs doublet theory are mentioned as a candidate theory to resolve the dark matter challenges in chapter~\ref{ch:dm}. 
Then, the CERN Large Hadron Collider and ATLAS detector are mentioned in detail as an experimental solution to examine new particle physics theories in chapter~\ref{ch:lhc} and chapter~\ref{ch:atlas}. 
The physics analytic workflow and statistical interpretation are demonstrated from chapter~\ref{ch:ana-intro} to chapter~\ref{ch:res}. 
Finally, the conclusion is shown in chapter~\ref{ch:con}.
